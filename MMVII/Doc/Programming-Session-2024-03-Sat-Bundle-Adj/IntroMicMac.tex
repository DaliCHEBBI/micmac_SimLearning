\documentclass[notes]{beamer}
\usetheme{Madrid}
\usepackage[utf8]{inputenc}
\usepackage{amsmath}
\usepackage{amsfonts}
\usepackage{amssymb}
\newcounter{saveenumi}
\newcommand{\seti}{\setcounter{saveenumi}{\value{enumi}}}
\newcommand{\conti}{\setcounter{enumi}{\value{saveenumi}}}

\resetcounteronoverlays{saveenumi}
%\author{}

\usepackage{geometry} 
\usepackage{array}
\usepackage{float}
\usepackage{placeins}
\usepackage{xcolor}
\usepackage{cancel}
\usepackage{graphicx}
\usepackage{tikz} 
\usepackage[overlay]{textpos}
\usepackage{hyperref}
\usepackage{animate}
\usepackage{wasysym}
\usepackage{listings}

 
 
\newcommand{\tikzmarkk}[2][minimum width=6cm,minimum height=1.5cm]{
 \tikz[remember picture,overlay]
 \node[anchor=west,
       inner sep=0pt,
       outer sep=6pt,
       xshift=-0.5em,
       yshift=-3ex,
       #1](#2){};
}

\newcommand{\shownode}[1]{
  \tikz[remember picture,overlay]\draw[red](#1.south east)rectangle(#1.north west);
}

\newcommand{\showanchor}[1]{
  \tikz[remember picture,overlay]\draw[red,thick,mark=x] plot coordinates{(#1)};
}

% original code from Stefan Kottwitz
% http://tex.stackexchange.com/a/12551/13304
\newenvironment<>{varblock}[2][.9\textwidth]{%
  \setlength{\textwidth}{#1}
  \begin{actionenv}#3%
    \def\insertblocktitle{#2}%
    \par%    
    \usebeamertemplate{block begin}%
    }
  {\par%
    \usebeamertemplate{block end}%
  \end{actionenv}}

\newenvironment{myblock}[1]{\begin{textblock*}{500pt}(#1)}{\end{textblock*}}

% special way to reach the top of the block
\def\newabove(#1){
([yshift=1.5ex]#1.center)
}

\definecolor{dgreen}{rgb}{0.,0.6,0.}
\colorlet{dgr}{green!70!black}
\definecolor{aqua}{rgb}{0.0, 0.2, 1.0}

 
\definecolor{persianindigo}{rgb}{0.2, 0.2, 0.7}  
\newcommand{\indigo}[1]{\textcolor{persianindigo}{#1}}

\definecolor{red}{rgb}{1.0, 0.0, 0.0}  
\newcommand{\red}[1]{\textcolor{red}{#1}}

\title{MicMac V1/V2 état d'avancement}
\subtitle{\textcolor{lightgray}{Ajustement de faisceaux}}
\author{Marc Pierrot Deseilligny 
%\footnotesize \textit{with support slides from A Pinte and M Pierrot Deseilligny}
} 

\institute
{
	Univ. Gustave Eiffel -- IGN/ENSG, LaSTIG lab.- France 
}

\date{novembre 2022}

\graphicspath{{./img/}}

%vertical curmy braces 
\usetikzlibrary{decorations.pathreplacing,calc}
\newcommand{\tikzmark}[1]{\tikz[overlay,remember picture] \node (#1) {};}

\newcommand{\COM}[1]{}
  
\begin{document}

\begin{frame}
\titlepage
\end{frame}

\begin{frame}
\tableofcontents
\end{frame}

%%%%%%%%%%%%%%%%%%%%%%%%%%%%%%%%%%%%%%%%%%%%%%%%%%%%%%
%%   \section{MicMac V1 et V2}
%%%%%%%%%%%%%%%%%%%%%%%%%%%%%%%%%%%%%%%%%%%%%%%%%%%%%%

\begin{frame}
\section{MicMac V1 et V2}
\begin{center}
    {\bf {\Large MicMac V1 et V2}}
\end{center}
\end{frame}



  % -----------------------------------------------------------------
\subsection{Historique  V1}


  %   -  -  -  -  -  -  -  -  -  -  -  -  -  -  -  -  -  -  -  -  -  -  -  -

\begin{frame}{Quelques jalons(1)}
\begin{enumerate}
    \item[2003]  code d'appariement d'images pour les MNS/MNT : multi-échelle, multi-images ;
                 pour l'orientation utilise {\tt OriLib}, modèle {\tt Grille};
    \item[2005]  ajout de méthode d'orientation et d'ajustement pour la calibration géométrique des
                 caméras numériques du LoEMI;
    \item[2007]  intégration des point SIFT comme points homologues , pipeline d'orientation automatique
                 à partir d'un "paquet" d'images:
    \item[2008]  début d'une collaboration structurante avec le MAP-CNRS dans le domaine du patrimoine;
    \item[2009]  intégration des modèle fisheye (+- "tous" les modèles)
    \item[2010]  diffusion assez large de MicMac dans différentes communautés scientifiques (patrimoine, environnement)
\end{enumerate}
\end{frame}

  %   -  -  -  -  -  -  -  -  -  -  -  -  -  -  -  -  -  -  -  -  -  -  -  -
\begin{frame}{Quelques jalons(2)}
\begin{enumerate}
    \item[2011]  pipeline de calcul de modèles en "vrai 3D" en $100\%$ automatique ,  (i.e, pas   du $2.5$ d):
    \item[2012]  début d'une collaboration structurante en science de la terre avec l'IPGP;
    \item[2013]  portage sous Windows;
    \item[2015]  prise en compte de l'ajustement de faisceau sur des capteurs satellitte;
    \item[2018]  \dots décision de lancer une deuxième version $V_2$.
    \item[2020]  début de recherches utilisant le deep learning dans $V_1$;
\end{enumerate}
\end{frame}

  %   -  -  -  -  -  -  -  -  -  -  -  -  -  -  -  -  -  -  -  -  -  -  -  -
\begin{frame}{Bilan-1 Points positifs }
\begin{enumerate}
    \item  une chaine photogrammétrique libre open source complète  (satellite, aérien, terrestre)  
    \item  une chaine photogrammétrique centrée sur la métrologie, offrant un contrôle fin des étapes de calcul (!= boite noire);
    \item  un outil qui a été largement utilisé à l'IGN (calcul de MNS France par la production,  études à IGN-Espace, 
           apprentissage de la photogrammétrie à l'ENSG, prestation en photogra terrestre par le SGM, utilisation 
           au Matis/Lastig);
    \item  a été largement utilisé par des scientifiques, ingénieur \dots dans le patrimoine et l'environnement (2010-2016) ;
    \item  a été financé par plusieurs acteurs privés ou public ($4$ thèses industrielles, $1$ ANR, $1$ FUI, projet CNES-TOSCA).
\end{enumerate}
\end{frame}


  %   -  -  -  -  -  -  -  -  -  -  -  -  -  -  -  -  -  -  -  -  -  -  -  -
\begin{frame}{Bilan -2 , points negatifs}

Un outil qui s'est dévloppé depuis $15$ ans, sans réelle stratégie globale, et pour l'essentiel avec 
un seul programmeur pour le noyau. Conséquences :

\begin{enumerate}
    \item  une chaine sans interface et compliquée à utiliser : noms de commandes sans rapport avec ce qu'elles font,
           message d'erreur peu clairs

    \item un code peu ou mal documenté; pas de test unitaire, fonctionnel ; 

    \item des choix de conception complexes, notamment des optimisation (CPU, mémoire) moins justifiées avec le 
          matériel actuel; le gros de la librairie est orientée traitement d'image, alors que l'usage majoritaire est photogrammétrique;

    \item  utilise peu de librairies externes, beaucoup de "home made" sur des fonctionnalité courante (lecture tiff maison,
           certaines interface sur XLib);

    \item  des contribution externes réalisées par des contractuels, pas facile à maintenir;

    \item  une chaine "viellissante" à l'ère du deep-learning, big data .
\end{enumerate}

\end{frame}

  %   -  -  -  -  -  -  -  -  -  -  -  -  -  -  -  -  -  -  -  -  -  -  -  -
\begin{frame}{Décisions}

     MicMac V1 ne pourra pas évoluer sur le long terme :
      
\begin{enumerate}
    \item  soit on "abandonne" à moyen terme, en se limitant à une maintenance currative;
    \item  soit on fait une nouvelle version.
\end{enumerate}

\pause

\vline

    Il y a un interet a avoir une chaine photogrammetrique open source  complete développée à l'IGN
    En faisant une deuxième itération grace au recul, on espère reprendre les bonnes idées et éviter  les ecueil de V1.

\begin{enumerate}
    \item  on n'abandonne pas et on fait une nouvelle version!
\end{enumerate}

\end{frame}

  % -----------------------------------------------------------------
\subsection{Le "projet" MicMac V2}


  %   -  -  -  -  -  -  -  -  -  -  -  -  -  -  -  -  -  -  -  -  -  -  -  -
\begin{frame}{Ulisateurs  prioritaires}

Utilisateurs  prioritaires :

\begin{enumerate}
    \item  étudiants et enseignants en photogrammétrie, notamment ENSG;

    \item  chercheurs en STIC utilisant la photogrammétrie, notamment au LaSTIG, CEREMA, UGE \dots

    \item  expert/ingénieur en photogrammétrie terrestre, aérienne et spatiale, notamment à l'IGN
           et dans le cadre de collaborations formalisées(CERN).
\end{enumerate}

\vline

$\rightarrow$ Les utilisations plus "grands publics" pour avoir des modèles visuellement  plaisants
ne sont pas une priorité, mais pourront se développer dans le cadre de collaborations (ex
développement pour les archives dans le cadre du stage d'Alexane NGhien).

\end{frame}

  %   -  -  -  -  -  -  -  -  -  -  -  -  -  -  -  -  -  -  -  -  -  -  -  -
\begin{frame}{Les "interfaces"}

\begin{enumerate}
     \item {\bf programmeur}, au niveau du code lui-même pour les programmeurs, avec objectif d'avoir une qualité de code et
           de documentation le permettant;

     \item {\bf expert},  au niveau ligne de commandes : noms rationnels, complétions automatique, spécifications des
                    commandes (entrée, sortie, objectif);

     \item {\bf étudiant},  vCommand systématisée

     \item {\bf expert,étudiant},  binding python permettant un accès relativement simple, à "grain plus fin" que les commandes;

     \item {\bf "grand public"},  dans le cadre de collaboration extérieur, interface finalisée, contacts pris avec Map-CNRS et
          meshroom.

\end{enumerate}

\end{frame}

  %   -  -  -  -  -  -  -  -  -  -  -  -  -  -  -  -  -  -  -  -  -  -  -  -
\begin{frame}{Organisation : table rase \dots}

\begin{enumerate}
    \item  évolution progressive incompatible avec les ambition de correction des problèmes
    \item  on part "from scratch"  
\end{enumerate}

\pause
\vline
\dots ou presque, il y a provisoirement des liens avec V1 :

\begin{enumerate}
    \item  on link avec V1, qui est utilisé comme librairie externe pour quelques fonctionnalités avant
           que l'on  choisisse la "bonne" lib (essentiellement lecture/ecriture des images, 
           gestion des systèmes de projection);


    \item  on peut importer des données V1 au format V2 pour prototyper des chaines complètes ; par exemple
	    import d'orientation initiale,  calcul de points homoloques ;

    \item  on peut faire du copié-collé de code sur des petits morceaux autonomes (marginal).

\end{enumerate}

\end{frame}

  %   -  -  -  -  -  -  -  -  -  -  -  -  -  -  -  -  -  -  -  -  -  -  -  -
\begin{frame}{Organisation : équipe \dots}

Actuellement un peu plus de $2$  ETP . A gros traits :

\begin{enumerate}
   \item  Mehdi Daakir, dans le cadre du CERN, test Window, développemet de tests fonctionnels;
   \item  Celestin Huet, dans le cadre du CERN, développement spécifique, ajout de librairies externes;
   \item  Yann Meneroux, détection de cibles pour PRISMA
   \item  Christophe Meynard, informatique  (portage windows, git, vcommand et complétion), dérivée formelle (avec MPD);
   \item  Jean Michaël Muller, binding python, ajout de la topographie;
   \item  Marc Pierrot Deseilligny, développement du noyau;
   \item  Ewelina Rupnik, estimation de pose  initiale rapide et précise.
\end{enumerate}

Plus un ingénieur de recherche (Christian Staron) travaillant  dans {\tt MMVII} dans le
cadre d'une collaboration IGN/CNES-Tosca/IPGP.

\end{frame}

  %   -  -  -  -  -  -  -  -  -  -  -  -  -  -  -  -  -  -  -  -  -  -  -  -
\begin{frame}{Etat d'avancement, général}

\begin{enumerate}
    \item  noyau ($70\%$ ??) : classe pour les commandes, la gestion de chantier, la serialization,
           les dérivées partielles, l'optimisation non linéaire, 


    \item  portage windows, ok en interne, premier beta testeur CERN en janvier;

    \item  vCommand et complétion, premier proto,  en beta;  

    \item   binding python : premiers exercices avec les PPMD  en $2023$;

    \item  tests unitaires : $80\%$ 

    \item  tests fonctionnel  : $40\%$ 

    \item  documentation algorithmique ($60\%$ ??) de l'existant

    \item  documentation programmeur à haut niveau  ($50\%$ ??) de l'existant

    \item  documentation programmeur  dans le code  ($70\%$ ) .

\end{enumerate}
\end{frame}

  %   -  -  -  -  -  -  -  -  -  -  -  -  -  -  -  -  -  -  -  -  -  -  -  -
\begin{frame}{Etat d'avancement, thématique}

\begin{enumerate}
    \item  détection de cible codées, pour l'IGN et le CERN : $90\%$;

    \item  classe capteur avec perspective central quasi complet, incluant plusieur modèle 
           de fishey et de nombreux modèle de distorsion;

    \item  estimation de poses initiales quelques algorithmes : relèvement dans l'espace (calibré ou non),
           matrice essentielles;

    \item  ajustement de faisceau cas conique : assimilation de : points d'appuis, points de liaison,
           contraintes de bloc rigide,  inclinomètres ($70\%$) ; possibilité de contrainte
           dans les équations (i.e. optimisation sous contrainte stricte);

   \item   quelques développements prototype pour intégrer de la topo;

   \item  ajustement de faisceau cas spatial ($30-50\% ??$) : gestion des RPC, gestion des changement
	   de coordonnées, calcul de modèle ajusté purement $2d$ ; reste à faire :
          gestion de modèle initiaux  physique (grille ? modèle type "airbus") ,  gestion
          de corrections "physiques" $3d$ (nécessite discussion et collaboration avec le 
          service de l'information spatiale);

  \item  développement de surface (cas motivé par une application concrète) : quelque modèle d'égalisation
	  de la radiométrie (purement paramétrique).
\end{enumerate}
\end{frame}


  % -----------------------------------------------------------------
\subsection{Feuille de route}
  %   -  -  -  -  -  -  -  -  -  -  -  -  -  -  -  -  -  -  -  -  -  -  -  -
\begin{frame}{Pipeline photogrammétrique}

Le pipeline photogrammétrique est classiquement divisé en $3$ étapes photogrammétriques et le reste :

\begin{enumerate}
    \item  calcul de mesure images éparses : points homologues , points d'appuis (détection de cible)

    \item  calcul de la géométrie des caméras : estimation initiale (cas conique terrestre) et ajustement ;

    \item  calcul de la scène $3D$ : appariement dense contraint par la géométrie;

    \item  égalisation radiométrique ;

    \item  géneration du produit final : de nuage de points, de modèle numérique de surface,  d'ortho photo, de mesh texturé  \dots
\end{enumerate}

Le point le plus avancé est le calcul de la géométrie, les autres points sont inexistant ou à l'état d'ébauche.

\end{frame}

  %   -  -  -  -  -  -  -  -  -  -  -  -  -  -  -  -  -  -  -  -  -  -  -  -

\begin{frame}{"Stratégie"}

La ligne directrice est :

\begin{itemize}
      \item avoir prioritairement une solution  pour le calcul d'orientation 
            (coeur de métier IGN ) , éventuellement la topométrie ;

     \item intégrer d'abord des solutions libre externes pour le calcul d'appariement
           éparse et dense;  

     \item compléter avec des solution interne pour : (1)  être maitre de son destin
          (2) intégrer des résultat de recherches (par ex thèse de Mohamed Ali Chebbi);

     \item externe pour l'appariement éparse : SIFT + solutions basées sur l'apprentissage   + autre ?

     \item externe pour l'appariement dense, notamment des solution basée sur l'apprentissage fonctionnant en épipolaire
           (MMVII assurant toute la machinerie complémentaire en amont et en aval).
\end{itemize}
\end{frame}

%%%%%%%%%%%%%%%%%%%%%%%%%%%%%%%%%%%%%%%%%%%%%%%%%%%%%%
% \section{Organisation du  code existant}
%%%%%%%%%%%%%%%%%%%%%%%%%%%%%%%%%%%%%%%%%%%%%%%%%%%%%%

\begin{frame}
\section{Organisation du  code existant}
\begin{center}
     {\bf {\Large Organisation du  code existant}}
\end{center}
\end{frame}


  % -----------------------------------------------------------------

\begin{frame}{Outils utilisés}
\begin{itemize}
	\item  langage de développement principal {\tt C++},  version minimale {\tt C++17} 

	\item  compilateur supporté actuellement {\tt g++, clang, MSVC++}

        \item portage testé régulièrement sur Linux et Windows, ca marche pour l'instant sous MacOS;

        \item automatisation de la compilation par cmake (3.15 au moins);

        \item gestion de version par {\tt git} ;

	\item binding python (prototype) avec {\tt pybind-11};

	\item documentation  élémentaire avec  {\tt doxygen};

	\item documentation  plus "haut niveau" avec un document {\tt Latex}.

\end{itemize}
\end{frame}

  % -----------------------------------------------------------------

\begin{frame}{Bibliothèques Utilisées}

On essaye d'éviter les deux extrêmes : MicMac-V1 \emph{tout à la main}, outil comme Orpheo-too box :
\emph{reposer sur les épaules de géants} . Les deux posent des problèmes de maintenance  à long terme \dots
Bibliothèques utilisées :

\begin{itemize}
    \item Eigen : algèbre matricielle;
    \item pybind11 :  binding python;
    \item GDAL : lecture/écriture des images (en cours);
    \item PROJX : système de coordonnées (à venir/V1 pour l'instant);
    \item Xif... :  méta-données des images (à venir/V1 pour l'instant);
    \item  quelques code open source  composé de  fichier directement inclus  dans MMVII
	    (triangulation de delaunay Volodymyr Bilonenko-MIT, gestion du format ply Nick Sharp-MIT) ;
\end{itemize}

\end{frame}


  % -----------------------------------------------------------------

\begin{frame}{Bibliothèques non utilisées}

Peut-être utilisé plus tard :

\begin{itemize}
     \item CGAL pour la triangulation 2D-3D la tetrahdrization (mais problème
           de licence);
\end{itemize}


\emph {A priori} Non utilisé :

\begin{itemize}
    \item bibliothèque de traitement d'image

    \item boost (la librairie standard contient à peu près tout ce dont on a besoin);
\end{itemize}

\end{frame}


  % -----------------------------------------------------------------

\begin{frame}{Précautions logicielles(1)}

Tout en restant du code de recherche fait avec les moyen du bord, un
certain nombre de précaution sont prises  pour limiter les bugs  :

\begin{itemize}
   \item une variable d'environnement {\tt The\_MMVII\_DebugLevel} fixe
         le niveau de vérification;

   \item à son plus haut niveau : tous les accès aux images sont vérifiés
         (débordement d'indice, de valeur) , les division par $0$ , les racine carrés négatives \dots

    \item à son plus bas niveau seules les vérifications liées aux erreurs utilisateurs sont effectuées;

\end{itemize}

Pour la gestion mémoire les classes peuvent dériver de {\tt cMemCheck} , dans ce cas (si
on est en mode DEBUG):

\begin{itemize}
      \item une vérification empirirque de non débordement en écriture sera faire à la destruction de l'objet;

      \item une vérification de la libération sera faite à la fin du process.
\end{itemize}
\end{frame}


\begin{frame}{Précautions logicielles(2)}

Un programme , appelé abusivement {\tt BENCH}, effectue des
test unitaires assez "complets", par exemple :

\begin{itemize}
	\item {\tt  MMVII Bench 1}  lance les test sur toute les fonctionnalité;
	\item {\tt  MMVII Bench 200  PatBench=Geom}  lance $200$ fois les test sur les fonctionnalité
		géometrique, avec des configurations aléatoires différentes et \dots plante à l'itération $36$
\end{itemize}

Plusieurs  ensembles de commande s'executant consécutivement sont présents dans le dossier {\tt MMVII-UseCaseDataSet},
en plus d'être des exemples didactiques, ce sont des précurseurs de test fonctionnels.

\end{frame}

  % -----------------------------------------------------------------

\begin{frame}{Quelques conventions/habitudes}

\begin{itemize}
	\item  on essaye de réduire la taille des header et la taille du
		code compilé :  pas de \emph{grosse} méthode \emph{inline}, 
		instantiation explicite des \emph{grosse} méthodes template dans les cpp;

         \item pour les classes applications, le code de dérivation automatique, on 
		 passe par des allocateur qui évitent d'avoir à exporter tous les header 
		 (voir présentation suivantes);
	
	\item  les header qui contiennet \emph{beaucoup} de code  \emph{inline} s'appellent 
		généalement {\tt *Tpl*} ;

	\item  les données membre commencent généralement par {\tt m} et les variable par {\tt a},
	        les noms de classes par {\tt c}, les {\tt typedef} par {\tt t}, les classe enum par {\tt e},

	\item  header sont essentiellement tous à plat sous {\tt include}; 

	\item   quelques headers spécifiques à un ensemble de source sont au même niveau que les sources
	        (exemple {\tt src/Sensors/cExternalSensor.h});

	\item   tous les sources sont au même niveau {\tt src/AA/BB.ccp} sauf le {\tt src/main.cpp}.
\end{itemize}

\end{frame}

  % -----------------------------------------------------------------

\begin{frame}{Les formats MMVII (1) : dans la \emph{jungle} des données photogrammétriques}

Pas de standard permettant 
d'échanger simplement d'un logiciel à l'autre.  Rien qu'en interne IGN, au moins $3$
famille de format \dots 

Au début de {\tt MMVII}, réflexion sur l'utilisation du format {\tt kapture}, format ouvert
développé par {\tt Naver-labs}, ayant des passerelle vers un certain nombre de solutions
photogrammétrique.   Mais un format d'échange, n'est pas forcément un bon format de travail.
Format de travail doit pouvoir représenter finement toutes les étapes d'un calcul tel
que modélisées par le logiciel, peu de chance qu'il puisse être représenté par un
format standard.

Format interne spécifique {\tt MMVII} : les grande lignes seront décrites dans la suite;
permet pour chaque type de données de stocker autant d'état de calcul que nécessaire.

La capacité d'importer/exporter des données est considérée comme importante mais est
dissociée du format interne.

\end{frame}

  % -----------------------------------------------------------------

\begin{frame}{Interface avec les autres formats}

Les interfaces suivante existent ou sont prévues, essentiellement en import pour l'instant :

\begin{itemize}
    \item import de donnée  micmac-V1, V2 étant pour l'instant très incomplète;

    \item import de nombreuses type de donéees structurées sous forme \emph{BSV}  (blank separated value) ou CSV
	    avec de commandes comme {\tt ImportGCP,  ImportTiePMul , ImportOri};

    \item import d'une calibration  sans distorsion avec la commande {\tt EditCalcMTDI};

    \item import d'orientation RPC {\tt ImportInitExtSens } , autres format à venir ?

    \item import des données du SIA/SMD  (cas particulier du BSV)

    \item import spécifique du format {\tt ORGI} du {\tt SIV }.
\end{itemize}

A moyen terme, import/export vers les solution courrantes, sans doute en se basant sur 
une solution type {\tt kapture} (qui supporte  COLMAP, bundler, nvm, OpenMVG, OpenSfM).

\end{frame}

  % -----------------------------------------------------------------

\begin{frame}{Documentation}

Le point faible de tous les logiciels \dots Les ressources suivantes sont accessible :

\begin{itemize}
    \item documentation au plus près du code avec {\tt Doxygen} :  lancer
          {\tt doxygen Doxyfile}

    \item documentation \emph{classique} sous forme d'un pdf à compiler à parir du Latex
	  sous {\tt MMVII/Doc};  ambitionne d'être une doc programmeur + utilisateur +
          algorithmique (cours de photogra),  vaste programme \dots  sans doute un quart
          de ce qui devrait être écrit l'a été;

    \item  des cas d'usage sous {\tt MMVII/MMVII-UseCaseDataSet/};

    \item  retour des sessions de formation : document rédigé par un ou deux stagiaires,
	    vidéo.

\end{itemize}

\end{frame}


\end{document}




