

%   ------------------------------------------------------------------
%   ------------------------------------------------------------------
%                 Chapter set editing
%   ------------------------------------------------------------------
%   ------------------------------------------------------------------

\chapter{Exercices on image orientation}

%   ------------------------------------------------------------------
%   ------------------------------------------------------------------
%   ------------------------------------------------------------------

\section{Introduction}

\subsection{Données/programmes d'entrées}

\begin{itemize}
   \item Des images avec  des cibles,  
   \item Les commandes  pour extraire la position des cibles
   \item La position 3D des cibles (pas toutes car aussi utilisation en tant que points de liaison)
   \item les poses et les calibrations de certaines images
   \item des outils pour visualiser en 2D la position de cibles dans les images
   \item  ??? des outils pour visualiser en 3D la position de point / caméra dans des scènes => pas un priorité en 2023,
	   sans doute en 2024
\end{itemize}


Le tag {\tt \bf OPTIONAL} indique que suivant le temps disponible, la vitesse d'avancement  ...,
l'exercice pourra !

\begin{itemize}
   \item fait lors de la séance
   \item fait en préparation
   \item fourni par les prof sous forme python (que les élèves curieux puisse le lire)
   \item fourni par les prof sous forme de primitive MMVII
\end{itemize}

%   ------------------------------------------------------------------
%   ------------------------------------------------------------------
%   ------------------------------------------------------------------

\section{Formules d'images, cas sans distorsion}

      %   ------------------------------------------------------------------

\subsection{Formule directe}

En entrée :

\begin{itemize}
   \item une image,
   \item sa pose et sa calibration (juste PP + F)
   \item quelque cible en $3d$  (par ex $4$)
   \item la position $2d$ (controle numérique) de certains de ces points (par exe $2$ sur les $4$)
\end{itemize}


Objectif :
\begin{itemize}
   \item Ecrire la fonction de projection terrain image.  
   \item Sous forme de fonction isolée ? Ou sous forme de méthode dans une classe python ? A voir.
\end{itemize}


Sortie/validation :

\begin{itemize}
   \item visualisation des points projeté
   \item controle numérique sur certain  points
\end{itemize}
   
      %   ------------------------------------------------------------------

\subsection{Intersection de faisceau}

\label{TutoBundleIntersec}

{\tt \bf OPTIONAL}

En entrée :

\begin{itemize}
   \item plusieurs ensemble de droites ;
   \item certaine s'interescte parfaitement, d'autre non
   \item quelques résultats de pseudo intersection
\end{itemize}

Ecrire une fonction réalisant l'intersection de $N$ droite. Peut-être omis, car c'est assez
basique et purement géométrique. Si on a le temps : c'est un outil qui revient en permanence
en photogra, c'est l'occasion de mettre en pratique les moindes carrés.  

On peut aussi,de manière intermédaire :

\begin{itemize}
	\item fournir la complétion du repère orthonormé  (à partir de $u$ trouver $v,w$ tels que $u,v,w$ soit un RON);
	\item fournir le "trick" $\trans v v + \trans w w = Id - \trans u u$ qui permet de calculer directement la matric
		des moinde carrés:
\end{itemize}

En sortie :

\begin{itemize}
   \item la coordonées du points 
   \item la visualisation dans un viewer $3d$ 
\end{itemize}

      %   ------------------------------------------------------------------

\subsection{Formule Inverse}
\label{TutoFormIWoD}

En entrée :

\begin{itemize}
   \item plusieurs image, par exemple $I_1,I_2,I_3,I_4$, leurs poses et calibration sans distorsion;
   \item les coordonnées de plusieurs points vus sur $2$ ou $3$ de ces images;
   \item pour certains de ces points la position $3d$ et/ou la position $2$ dans les autres image
\end{itemize}

Objectif :
\begin{itemize}
   \item écrire la fonction de projection inverse qui donne le faisceau partir d'un point, de la pose et la calib
\end{itemize}

Sortie/validation :

\begin{itemize}
   \item visualisation 3D de plusieurs faisceaux
   \item comparaison avec une valeur numérique sur qq faisceau (le vecteur directeur de la droite par exemple)
   \item intersection de faisceaux et reprojection sur une nouvelle image pour un controle 2D
\end{itemize}


%   ------------------------------------------------------------------
%   ------------------------------------------------------------------
%   ------------------------------------------------------------------

\section{Estimation de poses avec 12 paramètres}

%   ------------------------------------------------------------------
\subsection{Estimation des $12$ paramètres}

L'estimation de pose par la méthode des $11$ paramètres n'est pas très utiles en pratique
en photogramétrie (on préfère le relèvement dans l'espace, bcp plus robuste dès qu'on est
calibré). Cependant :


\begin{itemize}
    \item c'est didactique au sens où les élèves peuvent le programmer eux-même (contrairement au 
	    relevemnt dans l'espace);

   \item avec les données de la mire CERN ca marche à peu près,

   \item c'est didactique aussi si on montre aussi des exemples dégénérés (cible dans le plan)

   \item Eventuellement, demander à JPP des données qui permettent d'illustrer sur un cas où c'est utile "pour de vrai".
\end{itemize}

En entrée :

\begin{itemize}
   \item des images de cibles codées ;
   \item la commande  de détection de cibles  (certaines seront non detectées)
   \item la position $2d$ dans des fichiers à part de certaines cibles volontairement non detectees
   \item les coordonnées $3d$ des cibles
   \item le module de binding MMVII permertant de lire les cibles $2d$ et $3d$ et de les fusionner
         (récupérer directement une liste de paire $2d$ / $ 3d$ pour une images données);
    \item quelques valeurs numériques de résultat (genre 1 centre, 1 focale);
\end{itemize}

Objectif :

\begin{itemize}
    \item  estimer par moindre carrés les $11/12$ paramètres 
    \item  estimer le centre de prise de vue en utilisant la formule~\ref{PoseUnCalEstimC}
\end{itemize}

Sortie/validation :

\begin{itemize}
    \item  comparer le centre de prise de vue avec des valeur numériques sur certaines poses;
    \item  tester la reprojection sur des points de controles (points non détectés);
\end{itemize}

      %   ------------------------------------------------------------------

\subsection{Des $12$ paramètres à une "vrai" pose}

\label{From12PtoPose}
En entrée :

\begin{itemize}
    \item les résultats précédents ($12$ paramètres + séparation du centre);
    \item un module pour faire la décomposition $RQ$ , soit directement issu de MMVII par
          binding, soit en appelant numpy (option plus couteuse pour les stagiair, pas mal de pb 
          de convention à régler, pas une option PPMD, mais peut-être dans un cour élargi);
    \item le binding pour créer des objet  camera MMVII et les sauvegarder.
\end{itemize}

Objectif :

\begin{itemize}
    \item obtenir des paramètres physiques de pose
    \item analyser comparer ces paramètres
    \item préparer l'ajustement de faisceau photogrammétrique dans le section suivante:
    \item écrire la fomule d'image avec cette calibration linéaire
\end{itemize}

Sortie/validation :

\begin{itemize}
    \item générer la visualisation des poses ?  faisable éventuellement
          avec Apericloud ? Ou un module python a écrire ?

    \item analyser comparer ces paramètres, variabilité des paramètres internes
    \item regarder ce que cela donne sur un cas dégénéré (focale vaut n'importe quoi)
    \item tester la formule d'image et la valider avec pt de controle
\end{itemize}

%   ------------------------------------------------------------------
%   ------------------------------------------------------------------
%   ------------------------------------------------------------------

\section{Formules d'image avec distorsion}


%   ------------------------------------------------------------------
\subsection{Implémenter une fonction de distorsion, et la formule d'image}

\label{TutoFormImD}

Objectif : comprendre completement comment opère une formule d'images realiste.
A priori on part sur le modele dit de Fraser ou  $3-1-1$ :
$3$ coefficient radiaux, $2$ coefficient linéaire, $2$ coefficient décentriques.

Entree , +ou- comme le cas avec distorsion :

\begin{itemize}
     \item des poses et une calibration , en "pièces détachées" (que les élèves n'aient
           pas acces à ce stade aux projection MMVII ...)
    \item des correspondance 2D-3D
    \item des images et des points 3D
    \item des pair de point 2D avant/apres distorsion pour tester les resultats
\end{itemize}

Objectif : implémenter la formule d'image avec distorsion;


Résultats/validation :

\begin{itemize}
   \item controle de point distordu 2D;
   \item visualisation des points projeté
   \item controle numérique sur certain  points
\end{itemize}

%   ------------------------------------------------------------------

\subsection{Inverser une distorsion}

Objectif : voir une/des méthode d'inversion de function smooth, et pouvoir prolonger
les TD jusque vers des "vraies" application de photogra.

Deux options possibles :

\begin{itemize}
    \item inversion iterative sur le schema $(Id+\delta)^{-1} \approx Id-\delta $
   \item  inversion paramétrique par moindres carrés
\end{itemize}

La première est sans doute plus raisonnable pour des PPMD compte tenu du temps limité. La deuxième
plus robuste et rapide, pourrait faire partie d'un court plus long.

En entrée :

\begin{itemize}
    \item une distorsion lisible par le binding
\end{itemize}

Objectif : écrire une fonction qui prenne en paramètre la distorsion 
et un point et renvoie sont inverse.


Sortie/ validation :
\begin{itemize}
	\item  verifier sur des point image $D D'(x) = D'(x) D(x) = x $
	\item  verifier que sur des points trop éloigné : la méthode itérative diverge,
		la méthode par moindre carrés n'est pas précise.
\end{itemize}

%   ------------------------------------------------------------------

\subsection{Formule Inverse}
\label{TutoFormIDist}

Plus ou moins la même chose que~\ref{TutoFormIWoD} mais avec prise en compte de la distorsion.



%   ------------------------------------------------------------------
\subsection{Ajustement de faisceaux}

\label{TutoBundleGCP}
En entrée :

\begin{itemize}
      \item  des points d'appuis et leur mesure 2d issus du programme de détection de cibles
      \item  des pose initiales issues de \ref{From12PtoPose}, 
      \item  calibration intiale issue  de \ref{From12PtoPose},  typiquement une focale
             et un PP median
      \item  la formule d'image issue de \ref{TutoFormImD}
      \item  l'optimiseur universel de python.
\end{itemize}

Objectif : faire un ajustement de faisceaux, à partir de pose et calibration ayant des
valeurs initiales et de points d'appuis. Vérifier  que l'on peut estimer complètement la
calibration de la caméra.


Sortie/validation :

\begin{itemize}
      \item  comparaison avec une ground truh  sur certains paramètres
	      (une des pose, la focale par exemple);

      \item  analyse des résidus

       \item on garde une pose qui ne participe pas à l'ajustement,
             celle suite est ensuite affinée avec la calibration fixe :
             pour voir la stabilité de la calibration interne?
\end{itemize}

%   ------------------------------------------------------------------
%   ------------------------------------------------------------------
%   ------------------------------------------------------------------

\section{Estimation relative de $2$ poses} 

Objectif tester les méthodes d'estimation de $2$ poses qui sont 
souvent le germe initial de toutes les méthode. Tester aussi sur un
exemple minimaliste l'ajustement de faisceau avec des points inconnus.


%   ------------------------------------------------------------------
\subsection{Estimation par matrice essentielle}

\label{TutoME}
En entrée :

\begin{itemize}
      \item  des points homologues entre deux images : points sans outlayers
              et sur une scène $3D$, 
      \item  la calibration des caméra
      \item  pour passer de la ME à la pose relative soit (1)  un module numpy de SVD
	      (2) le binding sur le programme MMVII qui passe de la ME
		a l'orientation relative  ; (1) est évidemment plus instructif
		mais il y a beaucoup de convention différente entre les deux ...
\end{itemize}

Objectif : calculer la pose relative.

Résultat / validation :

\begin{itemize}
	\item sur un couple d'image une ground-truth (soit donnée 
		par les enseignant, soit calculée par les stagaire si elle vient
		d'une scène avec cible)
      \item sur d'autre couple validation par les résidus.
\end{itemize}

%   ------------------------------------------------------------------
\subsection{Estimation par homographie sur scène plane}

Sans doute trop spécifique pour les PPMD ??  Même un peu border line pour un cour spécialisé ?


%   ------------------------------------------------------------------
\subsection{Ajustement de faisceau par point homologue}

Dans les chaines photogrammétrique actuelle, l'essentiel quand ce n'est la totalité des 
mesure se fait à partir de points homologue et l'ajustement de faisceau se fait
essentiellement dessus.  Il n'est pas possible de faire un traitement complet sur point:
l'optimiseur universel python ne peut pas traiter de grand volume de données, on ne peut pas 
utiliser le complément de schurr.

L'objectif est de faire un ajustement de faisceau sur point homologues dans les condition minimales .


En entrée :

\begin{itemize}
    \item 2 images 
    \item leur calibration issue de \ref{TutoBundleGCP}
    \item une pose initiale issue de  \ref{TutoME}, pas très précise ?
    \item des points homologues très précis (éventuellement 
	    les cibles codées dégradée en point homologue) et peu nombreux (une dizaine)
    \item l'optimiseur universel de python
\end{itemize}

Objectif : 
\begin{itemize}
    \item a partir des points homologues, des calibration des poses initiale,
          calculer les faisceaux en utilisant  \ref{TutoFormIDist}

     \item  à partir de \ref{TutoBundleIntersec}, calculer la position initiale des points $3D$

     \item maintenant écrir la function d'ajustement à optimiser : avec seulement $5$ inconnue
           pour une des pose.
\end{itemize}

Validation/résultats 

\begin{itemize}
    \item résidu 
    \item ground truth
\end{itemize}

%   ------------------------------------------------------------------
%   ------------------------------------------------------------------
%   ------------------------------------------------------------------

\section{Relèvement dans l'espace}

Le relèvment dans l'espace est sans doute la méthode la plus utilisée
pour construires des blocs photogrammétriques une fois que l'on a déjà au moins 
deux images orientées.

Le calcul qui conduit à l'éblissement d'une équation du $4eme$ degre
est un peu fastidieux et pas très pédagogique. Donc on le prendra 
en binding de MMVII. Pour le reste, on peut essayer de jouer avec.

%   ------------------------------------------------------------------

\subsection{Des profondeurs aux poses}

En entrée :

\begin{itemize}
     \item une calibration interne;
     \item des points $2D-3D$, dont un triplet utilisé pour le relèvement
     \item un méthode de $MMVII$ qui étant donné $3$ direction de rayon et $3$
          GCP  donne les triplet de profondeur compatibles
\end{itemize}

Objectif estimer la pose suivant la démarche suivante :

\begin{itemize}
    \item en utilsant \ref{TutoFormIDist} estimer les direction de bundle
    \item en utililant la function MMVII estimer les triplets de profondeur
    \item avec les bundle on a les coordonées caméra
    \item écrire la formule permettant de calculer la rotation envoyant un triangle sur un autre.
    \item comme on a plusieur pose, estimer la meilleure avec les autres correspondance $2D/3D$.
\end{itemize}

Résultats/validation :


\begin{itemize}
     \item ground truth sur un des exemple
     \item résidu des points complémentaires
\end{itemize}

%   ------------------------------------------------------------------

\subsection{Estimation robuste}

\label{TutoRansacResec}
L'objectif est à la fois de "jouer" avec Ransac dans un contexte photogrammétrique
et de voir un exemple réaliste d'estimation de pose.

En entrée :

\begin{itemize}
    \item   une caméra calibrée;
    \item   des couples $3D/2D$ dont une proportion importante ($50\%$ ?) est très fausse
    \item   la méthode MMVII d'estimation de pose
\end{itemize}

Objectif : trouver la pose avec une approche ransac.  Combinatoire sur des petits
jeux de données où l'on teste tout les triplets, randomisée sur des jeux trop grands.

Résultat validation 

\begin{itemize}
    \item   ground truth
    \item   résidus
\end{itemize}

%   ------------------------------------------------------------------
%   ------------------------------------------------------------------
%   ------------------------------------------------------------------
\section{Estimation globale d'un ensemble de pose}

L'objectif est sur un cas simple de faire construire par les 
elèves un bloc global de poses.  Et accessoirement de leur faire
prendre conscience de la dérive si on ne fait pas de compensation
\footnote{et même si on en fait, mais il ne faut pas leur dire(;-)}

En entrée :

\begin{itemize}
    \item   une calibration des caméras
    \item   un ordonancement des caméras (choix de la paire initiale, puis ordre des image à rajouter en indiquant sur quels
	    images elle s'appuient);
    \item   des points de liaison mutltiple avec une API pour accéder simplement
	    aux requetes necessaires (à affiner ...) 
\end{itemize}


Objectif , programmer l'estimation de pose globable sur le schéma itératif suivant :

\begin{itemize}
    \item   estimer la pose relative de la première paire avec \ref{TutoME}
    \item   pour chaque nouvelle pose, utiliser les point multiples entre celles ci et les pose déjà orientées pour :

     \begin{itemize}
             
         \item   estimer la position $3d$ de ces point dans le repère relatif
         \item   estimer la pose nouvelle avec \ref{TutoRansacResec}
     \end{itemize}
\end{itemize}

Résultat / Validation :


\begin{itemize}
    \item   Apericloud ?
\end{itemize}





