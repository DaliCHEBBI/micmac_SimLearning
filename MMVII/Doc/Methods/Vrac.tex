

\chapter{Several stuff unfinished}


%---------------------------------------------
%---------------------------------------------
%---------------------------------------------

%\section{Co-variance propagation}

%---------------------------------------------
%---------------------------------------------
%---------------------------------------------

\section{Bloc fusion}

Hypothesis :

\begin{itemize}
    \item Two bloc $B_1$ and $B_2$  (the method should work with $N$ blocs),
          each has been oriented and we have an unknown similitude to compute between $B_1$ and $B_2$

    \item $N\geq 2$ triangle between $B_1$ and $B_2$
\end{itemize}

We suppose that the rotation has already been solved, which is "easy" because for each triangle,
the $2$ edges between the two different blocs give an estimation.  For each triangle we have an estimation
and quality check ($3$ measures of rotation when $1$ is enough). 

We need to estimate the scale $\Lambda$ and translation $T$. 
Let $P^1$  be a point measured in $B_1$ and $P^2$ be the same point in $B_2$, 
we thus have :

\begin{equation}
	P_2 = \Lambda P_1 + T \label{Eq:Bloc12}
\end{equation}

For each triangle we compute the rotation going to $B_1$ and $B_2$, and we still
have an unknown scale and translation for aligning it on $B_2$.
Let $\lambda_i$ and $t_i$  be the unknown scale and offset of the triangle.
Name $a,b,c$ the three points of a triangle, and $P^a_2, P^b_2 $ their corresponding positions
in $B_2$, which leads to:

\begin{equation}
	P^a_2    = a \lambda_i + t_i     \label{Eq:Bloc:Tri2}
\end{equation}

Now we have two possibilities for each point, if it belongs to bloc $B_2$ we know $P^x_2$ from equation
\ref{Eq:Bloc:Tri2} and we can use it directly. Else, we combine~\ref{Eq:Bloc12} and~\ref{Eq:Bloc:Tri2}
to obtain:

\begin{equation}
	 \Lambda P^a_1 + T = a \lambda_i + t_i \label{Eq:Bloc:Tri1}
\end{equation}

Finally, for  $n$ triangles we solve a \emph{linear} system with $4\cdot(n+1)$ unknowns:
$(\Lambda,T,\lambda_i,t_i)$ and we have $9\cdot n$ equations (by the way, even if we have 
$9$ equation for $8$ unknowns, the system is unslovable for $n=1$, explain why).

Proposed method :

\begin{itemize}
    \item if there are two triangles use ~\ref{Eq:Bloc:Tri2} and ~\ref{Eq:Bloc:Tri1}
          to estimate $\Lambda,T$;

    \item if there are $N>2$ triangles use some ransac strategy by selecting a pair of triangles
          and use previous case to solve it;

     \item if there is $1$ triangle, give up for now;  maybe later, use homologous point between
	     blocs, and outside the triangle to fixe the scale ????

\end{itemize}







