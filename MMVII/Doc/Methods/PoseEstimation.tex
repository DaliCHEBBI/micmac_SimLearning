
\chapter{Pose estimation, elementary method}


%-----------------------------------------------------
%-----------------------------------------------------
%-----------------------------------------------------

\section{Introduction}

This chapter present the "elementary" method, that compute the pose of an image
from observations that can, typically, be tie points (relative pose) or 
ground control point (relative or absolute pose).

By elementary we means algorithms that compute directly a solution for a limited
(typically $2$ or $3$, a few with tomasi-kanade) number of image. These algorithm
are the "tacticle" part, the result of the elementary algorithm will be used as
elementary part of the puzzle by more global method that will try to have a "strategic"  view.


%-----------------------------------------------------
%-----------------------------------------------------
%-----------------------------------------------------

\section{Space resection, calibrated case}


    %-----------------------------------------------------
\subsection{Introduction}

We deal here will the following problem  we have :

\begin{itemize}
   \item a calibrated camera ;
   \item a set of point for which we know the  $3d$ word coordinates $W_k$ and their 
        $2d$ coordinate $p_k$ in a image acquired with this camera;
\end{itemize}

We want to extract the pose $R,C$ of the camera (Rotation,center) such that for every point
we have the usual projection equation:

\begin{equation}
       \mathcal I(\pi (R*(C-W_k))) = p_k \label{EQ:PROJ}
\end{equation}


Each equation~\ref{EQ:PROJ}  creates $2$ constraints : one on line and one on column.
Also the pose $R,C$ has $6$ degrees of freedom, so we can expect that the problem has
a finite number of solution when we have $3$ points (in general less than $3$ will create
infinite number and more than $3$ no solution).

So here we will deal specifically  with the computation of the $R,C$ satisfying
exactly the equation~\ref{EQ:PROJ}  from a set of $3$ correspondance. Other
chapter will discuss of how we use more (possibly many) correspondance for beign 
robust to outlier (for
example with Ransac) or beign more accurate in presence of "gaussian" noise (for 
example with leas-square like) .


    %-----------------------------------------------------

\subsection{Putting things in equation}

First, we remark that if we know the local coordinates $L_k$ of the points in the repair
of the camera, the problem becomes easy : find a translation-rotation such that  :

\begin{equation}
       R*(C-W_k) = L_k  ; k\in{1,2,3} \label{SpResecEQ:WL}
\end{equation}

For each pai





