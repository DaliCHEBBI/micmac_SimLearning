

%   ------------------------------------------------------------------
%   ------------------------------------------------------------------
%                 Chapter set editing
%   ------------------------------------------------------------------
%   ------------------------------------------------------------------

\chapter{Inroduction to programmer'side}



\section{Adding a new command (1)}

In this chapter we will see the main roadmap to follow for
adding a new command in \PPP.


    % = = = = = = = = = = = = = = = = = = = = = = =

\subsection{Heriting from  {\tt cMMVII\_Appli}}

A first and easy principle is "One command/One class" and  "this class must
inherit from {\tt cMMVII\_Appli}.
In our case the class corresponding to {\tt EditSet} is the class {\tt cAppli\_EditSet}
defined in file :

\begin{itemize}
   \item  {\tt \MMVIDIR/src/cMMVII\_CalcSet.cpp}
\end{itemize}
As {\tt cMMVII\_Appli} is pure virtual class, the concrete class must override
$3$ methods :

\begin{itemize}
   \item {\tt int Exe();} : this method execute the action of the command;

   \item {\tt cCollecSpecArg2007 \& ArgObl(cCollecSpecArg2007 \&)} this method communicate the specification
         of mandatory parameters;

   \item {\tt cCollecSpecArg2007 \& ArgOpt(cCollecSpecArg2007 \&)} this method communicate the specification
         of optional parameters;
\end{itemize}


    % = = = = = = = = = = = = = = = = = = = = = = =

\subsection{Link between name an class}

The first thing \PPP has to do is to create the object heriting
from {\tt cMMVII\_Appli} from the command name (here create a {\tt cAppli\_EditSet}
from  {\tt EditSet}). To avoid huge compilation this creation is done
without declaration of all the class in header; the philosophy is to have
"hidden"  derived class definition in ".cpp" files  and just export an allocator.
More precisely this is done via the class {\tt cSpecMMVII\_Appli} which is
the specification of an application, it contains :

\begin{itemize}
   \item an allocator function able to create a {\tt cMMVII\_Appli} from command line
         (see type {\tt tMMVII\_AppliAllocator}), here this is the function {\tt Alloc\_EditSet};

   \item the name of the command, here {\tt EditSet};

   \item the comment of the command;

   \item three vector of specification : features (what main group the command belongs to), type
         of input, type of output;  these specification are used in the {\tt help} command to 
         look for a given command (satisfying a requets of the user such that "which command deal with oriention
         and produce ply data?");

   \item the file where the spec is created (using macro {\tt \_\_FILE\_\_}) to help recovering file from 
         command name.
\end{itemize}

Once the spec is created, it must be added to vector containing all the 
specification, this is done simply by adding a line as 
\begin{itemize}
   \item {\tt  TheRes.push\_back(\&TheSpecEditSet);} 
\end{itemize}

in file {\tt src/Appli/cSpecMMVII\_Appli.cpp}.


    % = = = = = = = = = = = = = = = = = = = = = = =

\subsection{Specifying paramaters}

%% \subsubsection{Genarilities}

The specification of parameters is done by the methods {\tt ArgObl} and {\tt ArgOpt}.
They both return a {\tt cCollecSpecArg2007} with are an agregation of {\tt cSpecOneArg2007}.
A {\tt cSpecOneArg2007} contains the specification of one parameters, it is
a virtual class that contains :

\begin{itemize}
   \item  the variable that will be initialzed, this variable which can be of different type
          as it is contained in the derived classes;

   \item  a vector of predefined semantics, a predefined semantic is create from one
          enum {\tt eTA2007} and an optional additional string;
   \item  the comment associated to the parameter;
   \item  the name of the parameter (aways empty string {\tt ""} for mandatory parameters);
\end{itemize}

Let's make a brief comment with class {\tt EditSet}, first mandatory parameters :

{\small
\begin{verbatim}
cCollecSpecArg2007 & cAppli_EditSet::ArgObl(cCollecSpecArg2007 & anArgObl)
{
   return
      anArgObl
         << Arg2007(mXmlIn,"Full Name of Xml in/out",{eTA2007::FileDirProj})
         << Arg2007(mOp,"Operator in ("+StrAllVall<eOpAff>()+")" )
         << Arg2007(mPat,"Pattern or Xml for modifying",{{eTA2007::MPatIm,"0"}});
}
\end{verbatim}
}

Let's comment :

\begin{itemize}
   \item The function {\tt Arg2007} is template and adapt to the type of the
         l-value , here all the parameter are string , but could be {\tt int} \dots or
         any type pre-instatiated in {\tt cReadOneArgCL.cpp} using macrp {\tt MACRO\_INSTANTIATE\_ARG2007};

   \item the {\tt StrAllVall<eOpAff>()} function is used to generate the string of all valid operators;

   \item the first parameter will fix the project directory, we indicate this by a having
         the semantic {\tt  \{eTA2007::FileDirProj\}}

   \item the third parameter will be the first main set of file, with indicate it by
         the semantic {\tt {eTA2007::MPatIm,"0"}};
\end{itemize}

{\small
\begin{verbatim}
cCollecSpecArg2007 & cAppli_EditSet::ArgOpt(cCollecSpecArg2007 & anArgOpt)
{
  return
    anArgOpt
       << AOpt2007(mShow,"Show","Show detail of set before/after , (def) 0->none, (1) modif, (2) all",{})
       << AOpt2007(mXmlOut,"Out","Destination, def=Input, no save for " + MMVII_NONE,{});
}
\end{verbatim}
}

Here we use the template function {\tt AOpt2007},
it is used with a {\tt bool} (parameter {\tt mShow})
and a {\tt std::string} (parameter {\tt mXmlOut}).
We see also that there is one more parameter : the name of the 
parameter that MicMac user will see (here {\tt Show} or {\tt Out}).

    % = = = = = = = = = = = = = = = = = = = = = = =

\subsection{Standard access paramaters}

Many command have paramaters that have more or less the
same meaning. It is possible to avoid code redundancy
via standard acces function and/or use of predefined semantic.

\subsubsection{Reading set of name from file with {\tt  SetNameFromString}}
\subsubsection{Main sets with {\tt MainSetk}}


%% \subsubsection{Genarilities}

% -------------------
% -------------------
% -------------------

