

\chapter{Radiommetric organization}

%---------------------------------------------
%---------------------------------------------
%---------------------------------------------

\section{Generality}

%---------------------------------------------
\subsection{Modelization}

The functionnality implemented currently in {\tt MMVII} do not pretend to make physicall
modelization of radiometry with approach like BRDF or radiatif transfer.
They are mainly targeted to make empiricall image equalization, which
can be summarized as this :

\begin{itemize}
      \item  as input we have set of homologous point $P_k$ with
	      $P_k = \{ (r_1 p_1) , (r_2 p_2) , \dots\}$ where $p_1,p_2 ...$
              are point seeing in image corresponding to the projection
              of the samepoint of the scene and $r_1, r_2 ...$ their 
              radiometry, it can be rgb or grey values;
             
      \item  a mathematical model of correction, that define for each image
	      the correction to recover the "real" radiometry  $G_k$;
\end{itemize}

The image equalizatio try to compute a set of model for $G_k$ such that
for each homologous point we have :

\begin{equation}
	G_i(r_i,p_i) = G_j(r_j,p_j)  \forall i,j \label{Rad:Eq:Equal}
\end{equation}

To avoid some trivial solution as $G_i=0$, that perfectly fit equation ~\ref{Rad:Eq:Equal}
we will often have to add some constraint as :

\begin{equation}
	G_i(r_i,p_i) \approx  r_i
\end{equation}

%---------------------------------------------
\subsection{Simplification}

In the current version of {\tt MMVII} we have made the following hypothesis :

\begin{itemize}
        \item  we modelize only grey-level correction;  the correction can be used on
		RGB image, but it will be the same on all chanel;
	\item  the correction is multiplicative   $G_i(r,p) = \frac{r}{F_i(p)} $
        \item  the correction $F$ is made from a sensor correction $S$ and per image correction $f_i$,
		$F_i(p) = S(p) f_i(p)$ ;
        \item  the sensor correction is a radial function, of center equal to principal point;
	\item  the per image correction $f_i(p)$ is a low polynomial function (classical choice
		for modelizing universal smooth function).
\end{itemize}

A probable next (and easy ?)  extension will add an additive term to the sensor 
correction to modelize the "dark current".

The choice $S$ is radial, and $f_i=Cste$, is a basic physicall modelization, the radial
function modelize the vignetage of lenses, while the constant modelize the global
energy conversion on each image (iso, speed ...).

In the default parametrisation we have :

\begin{itemize}
        \item one sensor model for each camera and each aperture;
	\item one per image correction $f_i$ per image (is it a totology to say that?).
\end{itemize}

%---------------------------------------------
\subsection{Main classes}

%---------------------------------------------
%---------------------------------------------
%---------------------------------------------

\section{Classes for photogrammetric data}
