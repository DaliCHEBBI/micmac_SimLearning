

\chapter{Classes for images}


%---------------------------------------------
%---------------------------------------------
%---------------------------------------------

\section{Introduction}

Also \PPP is not specificamly an image processing library, there are several part where
image are used to extract information and, at the end, image processing is a non neglectable part
of the code.
This chapter describe the general organization of image processing code.

By some historical language abuse we call \emph{image} what is now commonly called a tensor,
i.e. an array of $S_1 \times S_2 \times \dots S_d$ values , where $d$ is the dimension
of the image, $S_1$, $S_2$ \dots the sizes in each dimension.



%---------------------------------------------

\section{Files organization }

We begin by a description of files and folder related to images.

       %  - - - - - - - - - - - - - - -

\subsection{Connected files}

Note the following files that are not directly image files, but are strongly connected :

\begin{itemize}
	\item {\tt MMVII\_Ptxd.h} contain the definition of points, and consquently of pixels (point
		with integers coordinates),  contains the definition of boxes (and images inherit
		of boxes);

	\item {\tt MMVII\_Matrix.h} dense vector and dense matrix can be regarded as images ($1$ or $2$ dim), so in \PPP
		they are implemented as a shell arround  images of dimension $1$ or $2$


\end{itemize}

       %  - - - - - - - - - - - - - - -

\subsection{Header files}


The file containing class representing images are essentially :

\begin{itemize}
	\item {\tt MMVII\_Images.h} this file contain the classes for representing 
		generic images indepently of their dimension;
		it contains also the classes specific to images of dimension $1$ and $3$,
		which play a minor role:

	\item {\tt MMVII\_Image2D.h} this file contain the classes specific to $2$-dimensionnal
              images which obviously are the more frequent and more devlopped.
\end{itemize}


The definition (ie the code) of image processing routines can be found in "cpp" files
(for "big" routines) or inlined in header files (for "small" routines).
The declaration of routines defined in cpp files are found in :

\begin{itemize}
	\item {\tt MMVII\_Linear2DFiltering.h} contain linear filters , they are essentially
		gaussian filters routines and class for gaussian pyramid (for sift-like multiscale);

	\item {\tt MMVII\_NonLinear2DFiltering.h} contains declaration of non linear filters (for
		now their implemantation will often use  V1);

	\item {\tt MMVII\_ImageInfoExtract.h} contains declaration of classes and routines
		for information extraction, for now contains fast extremum extraction,
		and routine for extracting connected component on black and white image

	\item {\tt MMVII\_ExtractLines.h} contains declaration of classes and routine for 
		line extraction (hough transform);
\end{itemize}


The following files contains direcly the code of the function they implemant :

\begin{itemize}
	\item {\tt MMVII\_TplGradImFilter.h} contains code for gradient with optimization
		(by tabulation) for fast computation of polar decomposition 

        \item {\tt MMVII\_TplImage\_PtsFromValue.h} contain a very specific function for extracting
	      a point having a given value in an image (used for target "fine" detection);

      \item {\tt MMVII\_Tpl\_Images.h} containd codes for global basic operation on images not
	      related to spatial organization  (i.e operation that can be defined using just
		a vector of values); as they are not using spatial relation it contains also matrix
		operation; example  difference, sum, conversion , multiplication by constant,  reduction
		(sum of elements, bounds); 

	\item {\tt MMVII\_TplSimpleOperator.h} a tentative to implemant a library similar to V1 for genericity,
		but using templatization rather than virtualization; will see if it is devloped ...
	
	\item {\tt MMVII\_TplSymbImage.h} used when image are used in non linear optimization, see \ref{ImageOptDiff}

\end{itemize}

       %  - - - - - - - - - - - - - - -

\subsection{Cpp files}

The {\tt cpp} files can be found in the following folders :

\begin{itemize}
     \item {\tt ImagesBase/} contains the definition of image classe, it correspond to 
	     essentially to declaration of {\tt MMVII\_Images.h} and {\tt MMVII\_Image2D.h} 

     \item {\tt ImagesFiltrLinear/} contains linear filtering definition , correpond to declaration of
	     {\tt MMVII\_Linear2DFiltering.h}, and also some final command (to move ?);

     \item {\tt ImagesInfoExtract/}  contains the code definition for \emph{low level} "object" extraction
\end{itemize}

%---------------------------------------------
%---------------------------------------------
%---------------------------------------------

\section{Numerical types}

\label{NumericalType}

In {\tt MMVII} the following numerical types are defined in  {\tt MMVII\_AllClassDeclare.h}


\begin{itemize}
    \item {\tt tREAL4}, {\tt tREAL8} , {\tt tREAL16} for floatting point values on
	    $4,8$ and $16$ bytes;

    \item {\tt tINT1}, {\tt tINT2} , {\tt  tINT4}, {\tt  tINT8} for signed integer types
	    on $1,2,4$ and $8$ bytes;

    \item {\tt tU\_INT1}, {\tt tU\_INT2} , {\tt  tU\_INT4}, {\tt  tU\_INT8} for signed integer types
            on $1,2,4$ and $8$ bytes;
\end{itemize}



%---------------------------------------------
%---------------------------------------------
%---------------------------------------------

\section{General image organization}

\subsection{Parameter of image classes}

A class of image is parametrized by $2$ value :

\begin{itemize}
   \item the type on which is code each elements , it must be one of the numerical types
	   defined in \label{NumericalType}

   \item the dimension of the image,  the value can be $1$, $2$ or $3$ (there exist
	   also  unlimited dimension image, see ~\ref{UlimDimIm}  , but less efficient);
\end{itemize}


For maximal efficient manipulation, the code must "know" the type and the dimension
of the image.  For example if we manipulate $2d$ images with integer element 
include in $[0,255]$: , we will use images of type {\tt cDataIm2D<tU\_INT1>},
and this class will contain a field {\tt mRawData} of type {\tt tU\_INT1}
such that the reference to the value of $1$ pixel can be extracted by   {\tt mRawData[y][x]}.

Also some time it is interesting for limitating the size of code to maninuplate  images
independantly of their dimension or type. Sometime because this can be done as efficiently
as when we now the type and dimension, or sometime because effficiency is just not our priority.





% cDataGenUnTypedIm
% cDataTypedIm : public cDataGenUnTypedIm<Dim>
%template <class Type>  class cDataIm2D  : public cDataTypedIm<Type,2>

%---------------------------------------------
%---------------------------------------------
%---------------------------------------------

\section{Unlimited dimension image}
\label{UlimDimIm}

Classe {\tt cDataGenDimTypedIm}



