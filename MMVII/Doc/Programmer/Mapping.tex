

\chapter{The mapping object}

%---------------------------------------------
%---------------------------------------------
%---------------------------------------------

\section{Introduction}

%---------------------------------------------
%\subsection{Target of mapping objects}

The mapping  are targeted to offer service for object that represent "smooth" mapping
from $\RR^n  \rightarrow  \RR^p$. As an exemple of class naturally derived from mappings used
in photogrammetry we have :

\begin{itemize}
	\item projection $\pi : (x,y,z) \rightarrow (i,j)$ function of an image sensor, as mapping $\RR^3 \rightarrow \RR^2$;

	\item extended projection $\pi_d :  (x,y,z) \leftrightarrow (i,j,d)$ , where $d$ is  the depth,
		as \emph{bijective} mapping of $\RR^3$  ( $\RR^3 \rightarrow  \RR^3$);

	\item distorsion of central perpective camera as  \emph{bijective} mapping of $\RR^2$;

	\item any  transformation  between two geodetic coordinate systems as \emph{bijective}  of  $\RR^3$.

\end{itemize}

The minimal service that a mapping must offer is to define the method $F$ that computes its values.
The kind of services that offers the  mappingi package is :

\begin{itemize}
     \item offer a default method computing the derivative $\frac{\partial F}{\partial x_i}$  using a basic finite differrence ,
           the class can override this default method if has something better to offer;

   \item for $\RR^n \rightarrow \RR^n$ compute the inverse $F^{-1}(v)$ of a given  value using  an iterative method;
           the class can override this default method if has something better to offer;

   \item compute the approximate inverse mapping  of given mapping using some basis of function and a  least square approach;

   \item offer an interface to use generated code of symbolic derivative as a mapping.
\end{itemize}

%---------------------------------------------
%---------------------------------------------
%---------------------------------------------

\section{General organization}

\subsection{Localization}

The declaration of class for mapping are localized in file {\tt include/MMVII\_Mappings.h}.

The definition of these class are located in folder {\tt src/Mappings/}.
As the mapping class are template, there is an explicit instantiation  for
all expected use.



%---------------------------------------------
\subsection{class {\tt cDataMapping}}

         %  -  -  -  -  -  -  -  -  -  -  -  -  -  -
\subsubsection{Templatization}
The base  class of all mappings is {\tt cDataMapping}, its a template class defined by $3$ 
parameters :

\begin{itemize}
    \item {\tt class Type} which is the floatting number type on which all the computation will be made,
          it can be {\tt tREAL4, tREAL8} or {\tt tREAL16} ;  practically it is for now obly used
          with {\tt tREAL8}; by the way some precaution where made to assure that
          the class be intantiated with any complete numeric type in case higher precision woul be required;


    \item {\tt const int DimIn} the dimension of input space;

    \item {\tt const int DimOut} the dimension of output space.
\end{itemize}

         %  -  -  -  -  -  -  -  -  -  -  -  -  -  -
\subsubsection{Values}

The fundamental method that a  mappings must define is  {\tt Value(s)} and it computes the values of 
the function.  Note that there exist two methods :

\begin{itemize}
     \item {\tt Value} that make the computation of single value ;

     \item {\tt Values} that make the computation of vector of values, it can  be used
           if the class has some parallelism option to accelerate the computation.
\end{itemize}

Note that these two virtual methods  have a default implementation : {\tt Value}
is implemented calling {\tt Values},  while {\tt Values} is implemented calling
{\tt Value}.  So obviously, an infinite recursion will occur if none is defined
(BTW it is dynamically detected in debug mode).  The interest being obviously that
in the derived class, it's possible to overload only one to benefit of both.

For {\tt Value} there are two options : 

\begin{itemize}
     \item  one option where the user gives the vector for storing the result;

     \item  one option where the class furnish its own buffer by reference,
            btw the same vector is always returned, so if the user memorize
            the adress, at next call it will be overwritten; so if the vector
            is not used immediately, a copy must be made.
\end{itemize}



         %  -  -  -  -  -  -  -  -  -  -  -  -  -  -
\subsubsection{Jacobian}

The jacobian is computed by returning pair point/Matrix  where point is
the value  and matrix is the  jacobian. This is because generaly when the 
user needs the jacobian he also needs the value, and also when you 
compute the jacobian, you have also computed the value .

Be aware that even if the user make a copy a vector containing results,
due to {\tt MMVII} implementation of matrix (using shared pointer on data),
at next call the jacobian will overwrite the previous call.  In this rare
case, user should call the {\tt Dup} method.

Like {\tt Value} the  class propose default definition of {\tt Jacobian} that user can override.







