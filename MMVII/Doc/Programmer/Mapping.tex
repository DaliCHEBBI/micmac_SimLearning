

\chapter{The mapping object}

%---------------------------------------------
%---------------------------------------------
%---------------------------------------------

\section{Introduction}

%---------------------------------------------
%\subsection{Target of mapping objects}

The mapping  are targeted to offer service for object that represent "smooth" mapping
from $\RR^n  \rightarrow  \RR^p$. As an exemple of class naturally derived from mappings used
in photogrammetry we have :

\begin{itemize}
	\item projection $\pi : (x,y,z) \rightarrow (i,j)$ function of an image sensor, as mapping $\RR^3 \rightarrow \RR^2$;

	\item extended projection $\pi_d :  (x,y,z) \leftrightarrow (i,j,d)$ , where $d$ is  the depth,
		as \emph{bijective} mapping of $\RR^3$  ( $\RR^3 \rightarrow  \RR^3$);

	\item distorsion of central perpective camera as  \emph{bijective} mapping of $\RR^2$;

	\item any  transformation  between two geodetic coordinate systems as \emph{bijective}  of  $\RR^3$.

\end{itemize}

The minimal service that a mapping must offer is to define the method $F$ that computes its values.
The kind of services that offers the  mappingi package is :

\begin{itemize}
     \item offer a default method computing the derivative $\frac{\partial F}{\partial x_i}$  using a basic finite differrence ,
           the class can override this default method if has something better to offer;

   \item for $\RR^n \rightarrow \RR^n$ compute the inverse $F^{-1}(v)$ of a given  value using  an iterative method;
           the class can override this default method if has something better to offer;

   \item compute the approximate inverse mapping  of given mapping using some basis of function and a  least square approach;

   \item offer an interface to use generated code of symbolic derivative as a mapping.
\end{itemize}

%---------------------------------------------
%---------------------------------------------
%---------------------------------------------

\section{General organization}

\subsection{Localization}

The declaration of class for mapping are localized in file {\tt include/MMVII\_Mappings.h}.

The definition of these class are located in folder {\tt src/Mappings/}.
As the mapping class are template, there is an explicit instantiation  for
all expected use.



%---------------------------------------------
\subsection{class {\tt cDataMapping}}

         %  -  -  -  -  -  -  -  -  -  -  -  -  -  -
\subsubsection{Templatization}
The base  class of all mappings is {\tt cDataMapping}, its a template class defined by $3$ 
parameters :

\begin{itemize}
    \item {\tt class Type} which is the floatting number type on which all the computation will be made,
          it can be {\tt tREAL4, tREAL8} or {\tt tREAL16} ;  practically it is for now obly used
          with {\tt tREAL8}; by the way some precaution where made to assure that
          the class be intantiated with any complete numeric type in case higher precision woul be required;


    \item {\tt const int DimIn} the dimension of input space;

    \item {\tt const int DimOut} the dimension of output space.
\end{itemize}

         %  -  -  -  -  -  -  -  -  -  -  -  -  -  -
\subsubsection{Values}

The fundamental method that a  mappings must define is  {\tt Value(s)} and it computes the values of 
the function.  Note that there exist two methods :

\begin{itemize}
     \item {\tt Value} that make the computation of single value ;

     \item {\tt Values} that make the computation of vector of values, it can  be used
           if the class has some parallelism option to accelerate the computation.
\end{itemize}

Note that these two virtual methods  have a default implementation : {\tt Value}
is implemented calling {\tt Values},  while {\tt Values} is implemented calling
{\tt Value}.  So obviously, an infinite recursion will occur if none is defined
(BTW it is dynamically detected in debug mode).  The interest being obviously that
in the derived class, it's possible to overload only one to benefit of both.

For {\tt Values} there are two options : 

\begin{itemize}
     \item  one option where the user gives the vector for storing the result;

     \item  one option where the class furnish its own buffer by reference,
            btw the same vector is always returned, so if the user memorize
            the adress, at next call it will be overwritten; so if the vector
            is not used immediately, a copy must be made.
\end{itemize}



         %  -  -  -  -  -  -  -  -  -  -  -  -  -  -
\subsubsection{Jacobian}

The jacobian is computed by returning pair point/Matrix  where point is
the value  and matrix is the  jacobian. This is because generaly when the 
user needs the jacobian he also needs the value, and also when you 
compute the jacobian, you have also computed the value .

Be aware that even if the user make a copy a vector containing results,
due to {\tt MMVII} implementation of matrix (using shared pointer on data),
at next call the jacobian will overwrite the previous call.  In this rare
case, user should call the {\tt Dup} method.

Like {\tt Value} the  class propose default definition of {\tt Jacobian} that user can override.
The users has three option :

\begin{itemize}
      \item  furnish at the construction of object a "small" value that will be used to 
	      compute the derivative by finite difference;  this small value is a point as  the
             meaning of "small" can be different in each dimension;

      \item  furnish a version that compute jacobian for a single point;

      \item  furnish a version that compute jacobian for a vector of  points;
\end{itemize}

We have the same behaviour that with value : version with point, if not overided, calls version
with vector \dots 

Note that using Jacobian, it is possible to compute the affine mapping that is the
differential in a given point. This is done with {\tt Linearize}  (BTW, currently
this is done in derivate class {\tt cDataNxNMapping}, probably to change).

%---------------------------------------------
%---------------------------------------------
%---------------------------------------------

\section{Simple predefinite mappings}


%---------------------------------------------
\subsection{class {\tt cMappingIdentity}}

Sometime it can be usefull to desribe the identity function as a mapping.
For example as an estimation of rough inverse in iterative inversion.

%---------------------------------------------

\subsection{class {\tt cIMElemLinear}}
\subsection{class {\tt cBijAffMapElem}}
\subsection{class {\tt cInvertMappingFromElem}}

%---------------------------------------------
%---------------------------------------------
%---------------------------------------------

\section{Mapping inversion}

\subsection{Class {\tt cDataNxNMapping}}

This class is just a specializion for case where {\tt DimIn==DimOut}. All stuff
related to inversion will inherit from this class.


\subsection{Class {\tt cDataInvertibleMapping}}

This class is the mother class of mapping that are invertible. An invertible
mapping must simply define {\tt Inverse}  or {\tt Inverses} method that compute
the inverse of {\tt Value}.  Like usual, by default {\tt Inverse} calls  {\tt Inverses}
and {\tt Inverses} calls  {\tt Inverse}; so one at least must be defined else an
inifinite recursion will occur.


\subsection{Class {\tt cDataInvertOfMapping}}

When we have a mapping $M$ that is invertible, this class create a mapping $M'$
that contain a pointer on $M$ and such hat  $M'.Val(P) = M.Inv(P)$ and $M'.Inv(P) = M.Val(P)$


\subsection{Class {\tt cDataIterInvertMapping}}

This is here that we begin to really do something \dots

This class offer a concrete computation of the inverse function by iterative gradient method .
Given $Y$, we want to find $X$  such that $F(X)=Y$,  if we have a current "guess" $X_k$,
we write $F(X_{k+1}) = F(X_k) + \nabla (X_{k+1}-X_k) = Y$ then :

\begin{equation}
	X_{k+1} = X_k + \nabla ^{-1} (Y-F(X_k))
\end{equation}

This iterative schema require an initialization $X_0$, this is why the constructor take as
argument a mapping named {\tt RoughInv}. For the rough inverse sometime the user
will have information to give it, sometime identity (or even 
null function) can works well is the mapping is close to a linear form.  For more difficult case,
a convenient solution can be to use the facilities {\tt cComputeMapInverse} describes in~\ref{CompMapInv}
that compute approximate inverse by least square on a given basis of function (it's the method
used for invering distorsion).

\subsection{Class {\tt cDataIIMFromMap}}

This class is just an adaptator to use iterative inversion . Suppose class {\tt cMapSth} exist,
you want to invert it by iterative scheme of  {\tt cDataInvertOfMapping}, but if you cannot
modify {\tt cMapSth}, you cannot make it inherit of {\tt cDataInvertOfMapping}. So you can
use {\tt cDataIIMFromMap}, that owns a pointer on {\tt cMapSth} on inherit of {\tt cDataIterInvertMapping}
to make the inversion.

%---------------------------------------------
%---------------------------------------------
%---------------------------------------------

\section{Least square fitting, and use for inversion}

\section{class {\tt cLeastSqComputeMaps}}

This 


\section{class {\tt cComputeMapInverse}}

%---------------------------------------------
%---------------------------------------------
%---------------------------------------------

\section{Symbolic derivative \& mappings}

It's often convenient to encapsulate generated code in the class
mappings that offer a higher level description of the same code. 

cDataMapCalcSymbDer
cDataNxNMapCalcSymbDer
cLeastSqCompMapCalcSymb




\label{CompMapInv}






