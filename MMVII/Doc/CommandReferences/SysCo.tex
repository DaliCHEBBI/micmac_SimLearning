\chapter{SysCo manipulation}
\label{Chap:SysCo}


%-----------------------------------------------------------------------
%-----------------------------------------------------------------------
%-----------------------------------------------------------------------

\section{SysCo introduction}

{\tt MMVII} can import and export data from many coordinate systems (SysCo).

SysCo types supported by {\tt MMVII} are:
\begin{itemize}
\item \textbf{Local}: any euclidian frame, without any geolocalization or vertical knowledge.
\item \textbf{GeoC}: geocentric coordinates.
\item \textbf{LGeo}: a local euclidian frame with an affine transformation into geocentric coordinates.
\item \textbf{RTL}: a special case of LGeo where the local frame is defined by an origin point, where Z is ellipsoid normal at origin and X is East.
\item \textbf{Proj}: any georeferenced system handled by {\tt libproj} (including geographical coordinates).
\end{itemize}

When the SysCo is known or declared for an Ori or Measures folder, a file named {\tt CurSysCo.xml}
is created to record it.

\section{Setting SysCo}

\subsection{MMVII Commands}
To inform {\tt MMVII} of the SysCo of some data, there are several methods:

\begin{itemize}
\item Some importations have an implicit SysCo, E.G. {\tt ImportInitExtSens} that suppose that RPC are always in WGS geographical coordinates in degrees.
\item {\tt ImportOri} let you give the SysCo with {\tt SysCo} option.
\item {\tt ImportGCP}: let you transform ground coordinates on-the-fly with {\tt ChSys} option.
\item {\tt OriChSysCo} and {\tt GCPChSysCo}: let you transform Ori and ground points from one SysCo into an other.
\end{itemize}

\subsection{SysCo definition}
The SysCo names to give to {\tt MMVII} commands can be:
\begin{itemize}
\item The name of a file in source sub-folder {\tt MMVII/MMVII-RessourceDir/SysCo} or in project sub-folder {MMVII-PhgrProj/SysCo}, without its extension.
\item Any {\tt libproj} definition (such as {\tt IGNF:LAMB93} or {\tt EPSG:4326})
\item Any string starting with {\tt Local} for a local frame
\item Any string starting with {\tt GeoC} for geocentric
\item A string starting with {\tt LGeo} following the pattern:

{\tt LGeo*TX*TY*TZ*Omega*Phi*Kappa}, where the transformation is given in geocentric, the angles are in rad. 
\item A string starting with {\tt RTL} following the pattern: {\tt RTL*X0*Y0*Z0*Def}

(such as {\tt RTL*0.67451979*45.18899334*0.00000000*EPSG:4326}),
where you give the coordinates in a certain system of the tengance point of the local frame. Tip: use {\tt SysCoCreateRTL} command to make it automatically (see ~\ref{SysCoRTL}).

\end{itemize}


\subsection{Examples}
\begin{itemize}
\item {\tt SysCo=L93} will set the SysCo to Lambert93 (IGNF:LAMB93), as definied in \\
{\tt MMVII/MMVII-RessourceDir/SysCo/L93.xml}.
\item {\tt SysCo=LocalPanel} will set the SysCo to a local system named "LocalPanel", that will not be convertible into any other SysCo.
\item {\tt SysCo=IGNF:LAMB93} will set the SysCo to Lambert93.
\item {\tt SysCo=RTL*0.67451979*45.18899334*0*EPSG:4326} will set the SysCo to a local euclidian frame where origin is $0.67451979, 45.18899334, 0$ in EPSG:4326.
\item {\tt SysCo=Toto} will use a project-defined SysCo if {MMVII-PhgrProj/SysCo/Toto.xml} exists. If not, "Toto" will be used as a libproj definition, and an error will occur.
\item {\tt SysCo=GeoC} will set the SysCo to geocentric coordinates.

\end{itemize}


\section{RTL SysCo}
\label{SysCoRTL}
{\tt MMVII} suppose that the coordinates used during bundle adjustment are euclidian.

To keep small coordinates, a Local tangent frame (RTL) can be defined.

{\tt SysCoCreateRTL} command do that from an Ori, using as tangence point the average of canera positions or a fixed point.

It creates a file with the chosen name in {MMVII-PhgrProj/SysCo/}.

Then every Ori and ground measures can be transformed into this RTL frame to be able to keep maximal precision during bundle adjustment.
